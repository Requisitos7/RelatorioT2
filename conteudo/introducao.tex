\section{Introdução}
	Esse documento tem como objetivo apresentar a execução do trabalho 2 da disciplina de Requisitos de Software e compará-la com o planejamento realizado no Trabalho 1, ressaltando pontos positivos e negativos do desenvolvimento.

	Será feita uma breve descrição neste relatório do contexto de negócio atual da ESN (\textit{AS-IS}) e a melhoria proposta (\textit{TO-BE}) dos processos internos da empresa. Também serão apresentados o problema encontrado e uma análise do mesmo. Serão ainda apresentados tema de investimento, épicos, features, histórias de usuários, histórias técnicas e requisitos não funcionais identificados.

	Serão descritos também os usuários identificados que interagem com a aplicação e uma apresentação da tabela WSJF que foi criada para priorização das features. Tendo essa identificação será apresentado o roadmap com as features divididas por releases.

	Serão apresentads também as técnicas de elicitação utilizadas pelo grupo e a descrição de detalhadamente de como foram aplicadas.

	Após todas as identificações será feita uma análise do rendimento e experiência da equipe no contexto do trabalho realizado e de toda a aprendizagem obtida com a disciplina.