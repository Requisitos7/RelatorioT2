\section{Técnicas de Elicitação}
	\subsection{Workshop}
		O primeiro contato com o cliente deu-se com a técnica de workshop. Para a realização da técnica foi feito o planejamento da reunião conforme descrito abaixo:

		\textbf{Facilitador: }Eduardo Moreira.\\
		\textbf{Participantes: }Equipe de Desenvolvimento (Bruno Bragança, Eduardo Moreira e Ricardo Lupiano), Equipe de Modelagem (Iago Rodrigues e Vinicius Borges) e Monitores (Attany e Ebenezer).\\
		\textbf{Local: }Faculdade Gama - FGA.\\
		\textbf{Data: }26 de maio de 2015.\\
		\textbf{Horário: }11 horas.\\
		\textbf{Duração Prevista: }Aproximadamente 2 horas.\\
		\textbf{Objetivos:}\\
		\begin{itemize}
		{
			\item Compreender o contexto de negócio;
			\item Identificação de tema de investimento;
			\item Identificação de épicos.
		}
		\end{itemize}

		Nesse planejamento constaram questões levantadas pela própria equipe responsável para servirem de guia para se chegar às respostas e se atingir os objetivos almejados para o \textit{workshop}. As questões a seguir tiveram essa função de guiar o evento.

		\begin{itemize}
		{
			\item Qual a área de atuação da empresa? (Esporte, Saúde, Serviço Público, etc).
			\item Como é a organização da empresa?
			\item Qual o tamanho da empresa? (micro-empresa, pequena, média, grande, etc).
			\item Quais são os objetivos da empresa? (Visão da empresa).
			\item Qual a quantidade atual de clientes?
			\item Qual o perfil do cliente que a empresa visa atender?
			\item Como se dá o contato com o cliente?
			\item Qual o problema atual que o cliente precisa resolver?
		}
		\end{itemize}

		Utilizando-se da estrutura de um \textit{Brainstorming}, foi designado ao integrante Ricardo Lupiano que desempenhasse o papel de Relator da reunião ao qual foi incumbida a função de realizar as anotações durante toda a reunião. Essas anotações foram organizadas em tópicos que estão a seguir.

		\begin{itemize}
		{
			\item A empresa possui 2 sócios, e teve até hoje 3 clientes (Bicos).
			\item É uma micro-empresa;
			\item Atua no ramo de criação de sites \textit{e-commerce}.
			\item Atua atendendo amigos, o indicações de amigos.
			\item Hoje na ESN não existe um foco em qual tipo de cliente investir, mas pretende obter este foco com o resultado da pesquisa de mercado.
			\item Tem problemas na área de \textit{marketing}.
			\item Tem problemas com as requisições de clientes.
			\item Cliente MPR deseja obter uma forma de realizar o processo de validação do roteiro de pesquisa mais rápida, paralelizando a validação com a criação da mesma.
			\item Cliente MPR deseja obter uma forma de se obter o resultado da pesquisa de mercado simultaneamente com a aplicação da mesma.
		}
		\end{itemize}

		Ao final do workshop puderam-se identificar o tema de investimento e um épico do projeto, porém, após realizar a validação destes com o professor, percebeu-se que os épicos deveriam ser divididos para facilitar a derivação das features.
		
		Nesse evento, apenas os objetivos levantados inicialmente foram cumpridos, a compreensão do contexto de negócio, o tema de investimento (Marketing) e um épico (Roteiro de Pesquisa). Lembrando que o tema de investimento e o épico foram redefinidos posteriormente.
		
		Durante a realização da técnica, foram levantados dois problemas encarados pela empresa no momento atual. O primeiro se refere à demora no processo de realização e validação do Roteiro de Pesquisa. Isso porque, como já foi dito neste trabalho, apenas após o término do Roteiro de Pesquisa o mesmo pode passar para o processo de validação. O outro problema identificado foi a carência de se ter acesso ao Relatório de Pesquisa de Mercado durante o processo, ou seja, acompanhar os resultados da pesquisa simultaneamente

	\subsection{Entrevista}

		Neste segundo encontro, alguns objetivos importantes foram alcançados. Além de haver um melhor entendimento do contexto, as ideias e concepções de tema de investimento e épicos foram consolidados. Outro objetivo alcançado nesse evento foi a identificação das features correspondentes a cada épico.
		
		A seguir, encontram-se tópicos que marcam aspectos logísticos do evento realizada.

		\noindent
		\textbf{Facilitador: }Eduardo Moreira.\\
		\textbf{Participantes: }Equipe de Desenvolvimento (Bruno Bragança, Eduardo Moreira, Omar Faria e Ricardo Lupiano), Equipe de Modelagem (Jonathan Moraes) e Monitores (Attany).\\
		\textbf{Local: }Faculdade Gama - FGA.\\
		\textbf{Data: }2 de junho de 2015.\\
		\textbf{Horário: }12 horas e 30 minutos.\\
		\textbf{Duração Prevista: }Aproximadamente 2 horas e 30 minutos.\\
		\textbf{Objetivos:}\\
		\begin{itemize}
		{
			\item Compreender o contexto de negócio;
			\item Consolidar tema de investimento;
			\item Consolidar épicos.
			\item Identificar \textit{features}.
		}
		\end{itemize}

		\noindent
		Questões
		\begin{itemize}
		{
			\item Quais os atores envolvidos no processo de Pesquisa de Mercado da empresa?
			\item O que é necessário para iniciar o processo de Pesquisa de Mercado da empresa?
		}
		\end{itemize}

		Durante a entrevista foram identificados três atores principais do processo de Pesquisa de Mercado: Especialista de Marketing, Especialista de Mercado e Cliente Voluntário.

	\subsection{Prototipação}

		Essa técnica de elicitação foi realizada pelo time de desenvolvimento em conjunto com o cliente no momento da identificação das histórias de usuário. Foram realizados protótipos de papel para a melhor compreensão das histórias que geraram alguma dúvida ou mal compreensão por parte da equipe de requisitos. Através do protótipo foi possível a total compreensão por ambas as partes e dessa forma as histórias foram documentadas. Por se tratar de um protótipo de papel, esses foram descartados ao final da reunião.
