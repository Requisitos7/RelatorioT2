\section{Definições, acrônimos e abreviações}

Durante o processo de elicitação e gerenciamento de requisitos é necessário que todos os envolvidos possam se comunicar sem que existam falhas de entendimento, para isso, foi desenvolvido um sumário contendo nomes que serão utilizados no processo, assim como suas definições.

\begin{itemize}

	\item \textit{Histórias de Usuário}

		São chamados de requisitos funcionais todos aqueles que apresentam as funciona-lidades do sistema, tendo o mínimo de abstração possível (SOMMERVILLE, 2003).

	\item \textit{Feature}

		Features são funcionalidades apresentadas pelo sistema para que realizem uma oumais necessidade do cliente. Eles são mantidos noBacklogde Programa e prioriza-dos no SAFe através da tabela WSJF (Weighted Shortest Job First) (SAFe, 2015).

	\item \textit{Épico}

		Épicos de negócio são grandes iniciativas voltadas para o cliente que encapsulam odesenvolvimento necessário para prover certos benefícios comerciais.  Estão conti-dos e gerenciados no Backlog de Portifólio (SAFe, 2015).

	\item \textit{Tema de Investimento}

		Temas de investimento são objetivos que a organização define para a tomada dedecisões no nível de portifólio (SAFe, 2015).

	\item \textit{Requisitos} 

		Engloba tudo que o sistema deve possuir para solucionar o problema em questão, desde funcionalidades do sistema até características que o sistema deve possuir.

	\item \textit{Requisitos Funcionais}

		São chamados de requisitos funcionais todos aqueles que apresentam as funcionalidades do sistema, tendo o mínimo de abstração possível (SOMMERVILLE, 2003).
		
	\item \textit{Requisitos não Funcionais}

		São chamados requisitos não funcionais todos aqueles que apresentam as características do sistema, incluindo compatibilidade, o tempo de resposta ou qualquer outra exigência que não inclua funcionalidades (SOMMERVILLE, 2003).

	\item \textit{Engenharia de Requisitos}

		Engenharia de Requisitos é um conceito que engloba todo um contexto de desenvolvimento de sistema que envolve elicitação de requisitos, negociação, verificação e validação, e documentação e gerência de requisitos para o desenvolvimento de um sistema computacional. O uso da palavra \textit{Engenharia} garante que técnicas sistematicas serão utilizadas para que os requisitos sejam completos, corretos e consistentes (ESPINDOLA, 2004). 

	\item \textit{Framework do problema}

		Consiste em uma técnica para organizar e auxiliar o entendimento do problema e apresentar aos stakeholders afetados o impacto gerado para o cliente e uma possível solução bem sucedida. A utilização do framework garante maior facilidade no entendimento do contexto do cliente.

 
	\item \textit{WorkShop}

		 Workshop é uma técnica de elicitação de requisitos na qual os partipantes discutem um problema em comum enquanto são aplicadas técnicas que ajudam em uma melhor identificação das necessidades do cliente e a melhoraram o rendimento das reuniões.

	\item \textit{Brainstorming}

		 Brainstorming é uma técnica de elicitação de requisitos que consiste em uma dinâmica de grupo para recolher ideias a respeito de um determinado assunto.

	\item \textit{Sprint}

		Representa o espaço de tempo no qual deverão ser realizadas atividades previamente estabelecidas para a resolução de um problema (LEFFINGWELL, 2010).

	\item \textit{Release}

		São entregas de código funcional, as quais são feitas por etapa, entregando pequenas partes do sistema (LEFFINGWELL, 2010)..

	\item \textit{ESN}

		Empresa SEM-NOME: organização cliente do projeto.

	\item \textit{DRP}

		Demanda do Roteiro de Pesquisa.

	\item \textit{PRP}

		Proposta do Roteiro de Pesquisa.

	\item \textit{QP}

		Questionário de Pesquisa.

\end{itemize}