\section{Problemas}

	Os problemas foram identificados na área de Pesquisa de Mercado da empresa, a partir da análise de que este processo demanda uma carga horária muito mais alta do que outras atividades da organização.

	Muitas tarefas da pesquisa de mercado da ESN são iniciadas apenas quando outra tarefa seja completamente finalizada, causando uma demora na realização do processo. Outro fator que contribui para uma duração maior da pesquisa de mercado é a realização manual das atividades, podendo ser solucionada com a automatização das tarefas.

	\vspace*{0.5cm}

	\begin{table}[htbp]
		\caption{Framework de Problema}
		\begin{tabular}{|l|p{13cm}|}
			\hline
			\textbf{O problema} & A pesquisa de mercado da ESN é realizada de maneira manual e sequencial. Exigindo que os envolvidos aguardem a finalização de uma tarefa para iniciar outra. \\ \hline

			\textbf{afeta} & o especialista de marketing e o especialista de mercado contratados pela ESN. \\ \hline

			\textbf{cujo impacto é} & a demora no processo de pesquisa de mercado e obtenção dos seus resultados. \\ \hline

			\textbf{Solução} & Um software que possibilite a interação entre o especialista de marketing e o especialista de mercado, de modo que eles possam desenvolver os tópicos do roteiro da pesquisa de mercado e o questionário de pesquisa paralelamente. Além disso, proporcionar a geração de relatórios estatísticos e semânticos das respostas da pesquisa de mercado continuamente. \\ \hline
		\end{tabular}
	\label{Framework de Problema}
	\end{table}