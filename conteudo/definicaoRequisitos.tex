\section{Definições de Requisitos Ágeis}
	
	\subsection{Levantamento de Requisitos}

		\subsubsection{Nível de Portfólio}

			De acordo com o SAFe, a visão de portfólio é responsável pelo alinhamento da estratégia de negócios da organização e intençõess de investimento. Nela consta os temas de investimentos, os épicos de negócio e de arquitetura (LEFFINGWELL, 2010). 

			\begin{itemize}
			{
				\item Tema de Investimento:
				\begin{itemize}
				{
					\item TI: Pesquisa de Mercado\\
					Referência: Processo TO-BE, PRoblemas.
				}
				\end{itemize}

				\item Épicos:
				\begin{itemize}
				{
					\item E-1: Roteiro de Pesquisa\\
					Referência: Tema de Investimento.

					\item E-2: Aplicação de QP\\
					Referência: Tema de Investimento.

					\item E-3: Controle de Usuários\\
					Referência: Tema de Investimento.
				}
				\end{itemize}
			}
			\end{itemize}


		\subsubsection{Nível de Programa}

			O nível de programa é responsável pelo gerenciamento das releases e dos recursos, identificação e priorização das features, visando entrega contínua de valor para o cliente. (LEFFINGWELL, 2010).

			\begin{itemize}
			{
				\item Features:
				\begin{itemize}
				{
					\item FEAT-1.1: Manutenção de DRP.\\
					Referência: E-1.

					\item FEAT-1.2: Manutenção de PRP.\\
					Referência: E-1.

					\item FEAT-1.3: Validação de PRP.\\
					Referência: E-1.

					\item FEAT-1.4: Manutenção de QP.\\
					Referência: E-1.

					\item FEAT-2.5: Resposta de QP.\\
					Referência: E-2.

					\item FEAT-2.6: Geração de Relatório Semântico.\\
					Referência: E-2.

					\item FEAT-2.7: Geração de Relatório Estatístico.\\
					Referência: E-2.

					\item FEAT-3.8: Manuteção de Usuários.\\
					Referência: E-3.
				}
				\end{itemize}
			}
			\end{itemize}

		\subsubsection{Nível de Time}

			No nível de time, as equipes são organizadas de acordo com suas competências e habilidades para atender às demandas do projeto. Ocorrerá a definição das histórias de usuários, tarefas e serão realizadas iterações para a entrega de software. (LEFFINGWELL, 2010).

			\begin{itemize}
			{
				\item Histórias de Usuário:
				\begin{itemize}
				{
					\item US-1.1.1: Eu, como respecialista de marketing, desejo criar uma DRP para o especialista de mercado desenvolver a PRP.\\
					Critérios de Aceitação:
						\begin{enumerate}
						{
							\item Deverá possuir o campo \"nome\";
							\item Deverá possuir o campo \"Descreção\".
						}
						\end{enumerate}
					Referência: E-1, FEAT-1.1.

					\item US-1.1.2: Eu, como especialista de marketing, desejo atribuir uma DRP à um especialista de mercado para que ele possa visualizar a DRP.\\
					Critérios de Aceitação:
						\begin{enumerate}
						{
							\item Deverá ser possível visualizar uma lista de especialistas de mercado disponíveis;
							\item Deverá ser possível selecionar um especialista de mercado como responsável por uma demanda de roteiro de pesquisa;
							\item O sistema deverá apresentar uma mensagem caso o especialista de mercado selecionado já esteja responsável por outra demanda de roteiro de pesquisa.
						}
						\end{enumerate}
					Referência: E-1, FEAT-1.1.

					\item US-1.1.3:  Eu, como especialista de marketing, desejo alterar uma DRP para poder atualizar suas informações.\\
					Critérios de Aceitação:
						\begin{enumerate}
						{
							\item Deverá ser possível cancelar a DRP;
							\item O sistema deverá apresentar uma menságem de confirmação de cancelamento da DRP;
							\item Deverá ser possível alterar o nome e a descrição da DRP;
							\item Deverá ser possível alterar o responsável pela DRP através de uma lista de Especialistas de mercado disponíveis;
							\item O sistema deverá apresentar uma menságem de confirmação de alteração do responsável pela DRP.
						}
						\end{enumerate}
					Referência: E-1, FEAT-1.1.

					\item US-1.1.4:  Eu, como especialista de marketing, desejo importar arquivos do projeto para que a DRP contenha informações do projeto.\\
					Critérios de Aceitação:
						\begin{enumerate}
						{
							\item O sistema deverá apresentar a opção de adicionar arquivos durante a criação e alteração de uma DRP;
							\item Deverá ser possível importar arquivos do tipo: Imagem(JPEG, PNG), documentos (.docx, .pdf, .xls) e arquivos (.zip, .rar);
							\item Caso um dos arquivos seja uma imagem, deve-se criar uma galeria de imagem para visualização;
							\item Caso um dos arquivos seja documento, o mesmo deve estar disponível para download.
							\item Os itens só poderão ser modificados se exigida a devida correção;
							\item Caso o arquivo seja do tipo .zip ou .rar, deve-se descompactá-lo para que o acesso ao conteúdo seja possível.
						}
						\end{enumerate}
					Referência: E-1, FEAT-1.1.

					\item US-1.2.5:  Eu, como especialista de mercado, desejo manter os tópicos da PRP para construir uma PRP.\\
					Critérios de Aceitação:
						\begin{enumerate}
						{
							\item Deverá possuir os campos: Descrição, categoria, observação e referências;
							\item Ao deletar um tópico, deve-se pedir confirmação ao usuário;
							\item Não pode-se alterar nem deletar tópicos aprovados.
						}
						\end{enumerate}
					Referência: E-1, FEAT-1.2.

					\item US-1.2.6:  Eu, como especialista de mercado, desejo referenciar tópicos para criar uma relação entre eles.\\
					Critérios de Aceitação:
						\begin{enumerate}
						{
							\item Durante a criação ou alteração de um tópico deve-se poder visualizar uma lista de tópicos da mesma categoria;
							\item Deve-se poder selecionar um ou mais tópicos da lista para referenciá-los entre si.
						}
						\end{enumerate}
					Referência: E-1, FEAT-1.2.

					\item US-1.2.7:  Eu, como especialista de mercado, desejo submeter o tópico da PRP para solicitar a aprovação do tópico da PRP.\\
					Critérios de Aceitação:
						\begin{enumerate}
						{
							\item Para poder submeter um tópico deve ser preenchido ao menos o descrição e categoria;
							\item Deve-se ter um aviso de confirmação;
							\item Deve-se congelar um tópico que já foi enviado;
							\item Ser possível visualizar o status do tópicos.
						}
						\end{enumerate}
					Referência: E-1, FEAT-1.2.

					\item US-1.3.8:  Eu, como especialista de marketing, desejo aprovar o tópico submetido da PRP para consolidar o RP.\\
					Critérios de Aceitação:
						\begin{enumerate}
						{
							\item Deve ser possível visualizar os tópicos submetidos;
							\item Deve ser possível escolher entre as opções de aprovar e reprovar um tópico de PRP;
							\item Ao se aprovar todos os tópicos deve-se mostrar um aviso ao usuário mostrando que o RP foi consolidado;
							\item Ao reprovar um tópico deve-se mostrar uma mensagem de confirmação ao usuário;
							\item Tópicos reprovados devem deixar de serem visualizados pelo EMA;
							\item Deve-se exigir um comentário de justificativa ao reprovar um tópico.
						}
						\end{enumerate}
					Referência: E-1, FEAT-1.2.
				}
				\end{itemize}
			}
			\end{itemize}

