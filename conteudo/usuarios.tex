\section{Usuários}
	Ao desenvolver um sistema a uma organização, é importante conhecer não apenas seus usuários, mas também o ambiente em que eles estão inseridos. Isso auxilia na identificação da gestão dessa organização, no entendimento de como será utilizado a aplicação pelos usuários e a identificar restrições ambientais. Neste tópico também pode constar um estudo sobre a duração de cada tarefa no processo anterior, se já existe aplicação em utilização atualmente e quais mudanças nesses ambientes a solução de software causará (LEFFINGWELL, 2010).

	\subsection{Usuários Envolvidos}
		Foram identificados cinco tipos de usuários que interagem com a aplicação. São eles: Especialista de Mercado, Especialista de Marketing, Entrevistado, Especialista Geral e WebMaster.


		\begin{itemize}
		{

			\item O \textbf{Especialista de Mercado} é responsável pela área de marketing, suas funções incluem manter a PRP (proposta de roteiro de pesquisa). É ele o responsável pela criação e modificação dos tópicos que contemplarão a PRP.
			
			\item O \textbf{Especialista de Marketing} é responsável por manter uma DRP (Demanda de Roteiro de Pesquisa), ou seja, se ele que vai dar início ao processo no sistema.	É responsável também por ajustar as configurações de aplicação do QP (Questionário de Pesquisa).
			
			\item O \textbf{Entrevistado} pode ser considerado qualquer pessoa ou empresa que tenha respondido um questionário.
			
			\item O \textbf{Especialista Geral} nada mais é do que o especialista de mercado e o especialista de marketing trabalhando em conjunto. Suas funções são, validar uma PRP (Proposta de Roteiro de Pesquisa), manter o QP (Questionário de Pesquisa), acompanhar a aplicação dos questionários de pesquisa e analisar seus resultados.
			
			\item O \textbf{WebMaster} é a pessoa da empresa que possui permissão para modificar permissões dos usuários do sistema e gerenciar todas as contas registradas.

		}
		\end{itemize}