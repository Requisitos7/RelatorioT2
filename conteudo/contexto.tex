\section{Contexto de Negócio}
	A ESN é uma micro-empresa que atua na área de \textit{e-commerce} para pequenas empresas. Para apresentar destaque sobre os concorrentes, a ESN implementa técnicas de SEO, técnica que facilita o aparecimento de novos sites em motores de busca. Também é fornecida consultoria de publicidade aos clientes.

	A empresa se organiza em três áreas:
	\begin{itemize}
	{
		\item Grupo de Desenvolvimento;
		\item Grupo de \textit{Marketing};
		\item Grupo de Atendimento ao Cliente;
	}
	\end{itemize}

	Ao ser contratada, a ESN fornece um site simples e padronizado ao cliente. Tendo acesso apenas ao painel de administração, o cliente tem a possibilidade de realizar ações como inserir e editar categorias, programações, acessar relatórios e gerenciar pedidos. Também é dada a opção ao cliente de customizar o layout através de templates.

	A empresa ESN encontra-se com problemas na área referente à Pesquisa de Mercado da empresa. Existe uma dependência entre atividades que tem demandado muito tempo e dificultado o segmento processo. Além dessa dependência há a lentidão na realização da pesquisa já que a mesma é manual, ou seja, pessoas vão de porta em porta para realizar a pesquisa.

	\subsection{Descrição do Processo de Negócio Atual (\textit{As-Is}).}

		O processo de Planejamento de \textit{Marketing}, trabalho que possui maior relevância a ESN, foi modelado com base nas atuais circunstâncias e atividades da empresa. O mesmo apresenta quatro sub-processos que são: Pesquisa de Mercado; Definição de Público-Alvo; Posicionamento do Produto; e, \textit{Marketing} Mix.

		A seguir estão contemplados os Planejamento de \textit{Marketing} desenvolvido atualmente pela empresa e o sub-processo de Pesquisa de Mercado o qual está.

		<Imagem aqui>


		O processo de criação do Roteiro de Pesquisa é atualmente realizado de maneira lenta, pois uma atividade só pode ter início após a finalização por completo da atividade anterior. Um exemplo disso é que atualmente, o processo de validação do Roteiro de Pesquisa só é iniciado depois que o mesmo é concluído.

		<2 imagens aqui>


	\subsection{Descrição do Processo de Negócio Futuro(\textit{TO-BE}).}

	O processo a seguir foi identificado a partir de um primeiro contato com o cliente em uma atividade de elicitação de requisitos e representa o processo alterado que será contemplado na aplicação a ser desenvolvida.

	Quatro papéis interagem com o processo de Pesquisa de Mercado, são eles: Equipe de Marketing, Equipe de Desenvolvimento de Software, Especialista de Mercado e Clientes Voluntários. O processo é iniciado quando ao surgir uma Necessidade de Pesquisa de Mercado a Equipe de Marketing define a espécie de Produto para a Pesquisa. Em seguida, três atividades são realizadas paralelamente. 

	O Especialista de Mercado classifica possíveis consumidores, a Equipe de Desenvolvimento de Software cria um protótipo para espécie de produto e a Equipe de Marketing valida o protótipo.

	Na próxima etapa do processo outras duas atividades são paralelizadas, o Especialista de Mercado é responsável por desenvolver o Roteiro de Pesquisa e a Equipe de Marketing por validá-lo. O Roteiro de Pesquisa é composto por diversos tópicos. Ao passo que o Especialista de Mercado vai criando cada tópico, os mesmos já chegam para a Equipe de Marketing para que a mesma possa validá-los. Dessa forma, o processo é agilizado.

	Após validado o Roteiro de Pesquisa, passa-se para a etapa de resposta de Pesquisa de Mercado realizada por Clientes Voluntários e gerenciada pela Equipe de Marketing. Caso as informações geradas pela Pesquisa de Mercado sejam satisfatórias o processo é finalizado, senão retorna-se para o início e é definida a Espécie de Produto para Pesquisa novamente.


	<Imagem>
