\section{Relato de experiênia da disciplina de Requisitos de Software}
	Na disciplina de Requisitos de Software foi possível constatar a importância da execução do processo de Engenharia de Requisitos dentro de um projeto de desenvolvimento de software. Aprofundamos o estudo sobre as 5 atividades fundamentais da Engenharia de Requisitos: Elicitação, Análise e Negociação, Documentação, Verificação \& Validação e Gerência de Requisitos.

	Foi possível estudar as técnicas de elicitação de requisitos e colocar algumas delas em prática. Aprender a identificar problemas e a conduzir uma conversa com o cliente para elicitar os requisitos de software. Diferenciar o que o cliente deseja do que ele realmente precisa, entendendo que o papel do engenheiro de requisitos é compreender as reais necessidades do cliente e propor uma solução de software que contemple as funcionalidades que irão solucionar os problemas enfrentados pelo cliente.

	O conteúdo em sala de aula foi abordado de uma maneira que os alunos pudessem imaginá-lo sendo aplicado em um contexto real, o que possibilitou um entendimento mais fácil da matéria e maior interesse para com a disciplina.