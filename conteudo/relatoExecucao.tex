\section{Relato da Experiência da Execução do Trabalho}
	Durante a execução do trabalho, foi possível constatar que conforme previsto, a equipe de requisitos se manteve unida e conseguiu conduzir o desenvolvimento conforme o planejado.Tivemos dificuldade de comunicação com o grupo de modelagem durante o desenvolvimento do trabalho 1. Depois do feedback obtido do trabalho 1, a equipe de requisitos começou a interagir constantemente com a equipe de modelagem, melhoramos nossa comunicação e, dessa forma, pudemos atingir bons resultados.

	A maior dificuldade da equipe foi em gerenciar os prazos das atividades e se manter dentro do cronograma. Porém, conforme as tarefas foram sendo concluídas, o cronograma foi sendo reajustado para que a equipe pudesse se reprogramar e conseguir cumprir com os objetivos do trabalho dentro do prazo estipulado.

	Outra dificuldade foi o entendimento do contexto do negócio através do workshop realizado, que devido a inexperiência da equipe em aplicações de técnicas de elicitação e devido também a um participante da equipe fazer parte simultaneamente da equipe de Requisitos e Modelagem, o entendimento do contexto de negócio ficou confuso. Porém, isso foi contornado pois esse participante parou de fazer parte de nossas reuniões e mantivemos um maior contato com o restante da equipe de modelagem, começando de novo o entendimento do negócio através de outra técnica de elicitação, que dessa vez foi a entrevista.

	As técnicas de elicitação foram de suma importância para a elicitação de requisitos nas diferentes etapas do projeto. A técnica de prototipação foi utilizada durante o encontro com a equipe de MPR para definição das Histórias de Usuários assim que surgia qualquer dúvida sobre a utilidade de determinada História de Usuário. Já a entrevista teve sua importância na definição das features do sistema. Porém, a técnica de workshop foi a mais interessante, pois, apesar dos integrantes não terem experiência com tal técnica ela se mostrou produtiva, esclarecendo as necessidades do cliente e a identificação dos problemas da empresa.
	
	Foi um trabalho que exigiu esforço constante dos integrantes ao longo do semestre, um integrante ajudando o outro e sempre buscando compartilhar conhecimento entre a equipe. Algumas atividades foram paralelizadas para que conseguíssemos ficar dentro do prazo estipulado e aumentar a produtividade da equipe.

	A forma como o projeto foi proposto proporcionou uma experiência gratificante para todos os integrantes do grupo 7 de Requisitos, pois interagimos com outra equipe como em um projeto real, o que envolve dificuldades externas. Essas diversidades foram, sem dúvidas, importantes no processo de aprendizagem e permitiram um 